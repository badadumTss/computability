\chapter{Algorithms and existence of non-computable functions}

\section{Characteristics of an algorithm}
\label{se:alg-char}

We present a list of features that an algorithm should satisfy in
order to capture the intuitive idea of effective procedure. Roughly,
what we will ask is that it is ``implementable'' on some sort of
idealised machine, the computational model. Hence, in turn, we will
list some requirements that the computational model should meet to be
considered effective.

An \textbf{algorithm} is a sequence of instructions with the following characteristics:
\begin{enumerate}[label=\alph*)]
\item
  \label{as:prog_fin}
  it is of \textbf{finite length};
\item there exists a \textbf{computing agent} able to execute its instructions;
\item the agent as a \textbf{memory} available (to store input, intermediate results to be used in subsequent steps and output)
\item the computation consists of discrete steps (it does not rely on analog devices)*
\item the computation is neither nondeterministic nor probabilistic (
  we model a  digital computer)
  
\item there must be no limit to the size of the input data\\
  (we want to be able to define algorithms that work on any possible
  input, e.g. sum \dots operating on summands of any size);
  
\item there is no limit to the memory that can be used\\
  This requirement could appear less natural, but having an unlimited
  memory is essential to avoid the the notion of computability depends
  on the specific resources which are available. In fact, for many
  functions the space required for the intermediate results depends on
  the size of the input.
  
  e.g. $f(n) = n^2$ $(1000)^2 = 1000000 \leftarrow$ I must add a
  number of zeroes that depends on $n \rightarrow n$ must be stored
  (the states are finite).

\item
  \label{as:istr_fin}
  there must exist a finite limit to the number/complexity of the
  instructions
  
  This is intended to capture the intrinsic finiteness of the
  calculation device (justified by Turing with the limits of the human
  mind/memory);
  
  E.g. for a computer, the memory that can be accessed with a single
  instruction must be finite (even if by
  (g), the memory is unlimited);
  
\item computations might
  \begin{enumerate}
    
  \item  end and return a result after a finite, but unlimited number of steps  
    (e.g. the square function requires a number of steps proportional to the argument);
    
  \item continue forever, and not return a result.
  \end{enumerate}
\end{enumerate}

\section{Existence of non-computable functions}

Later we will focus on a concrete computational model, that will allow
us to give a completely formal definition of computable function. Here
we observe that, quite interestingly, simply on the basis of the
assumptions above, we can infer the existence of non computable
functions for every ``effective'' computational model.

We start by recalling some basic notions and introducing useful
notation.

\begin{itemize}
\item We will consider the set of \emph{natural numbers}
  $\nat = \{ 0, 1, 2, \dots \}$;

\item Given the sets $A, B$ their \emph{cartesian product} is
  $A \times B = \{ (a,b) \mid a \in A\ \land\ b \in B\}$. We will
  write $A^n$ for $A \times A \times A \times \ldots \times A$ $n$
  times. More formally $A^1 = A$ and $ A^{n+1} = A \times A^n$.
  
\item A (binary) \emph{relation} or \emph{predicate} is
  $r \subseteq A \times B$.
  
\item A \emph{(partial) function} $f : A \to B$ is a special relation $f \subseteq A\times B$ such that if $(a, b_1), (a, b_2) \in f$ then  $b_1 = b_2$.  Following the standard convention, we will write $f(a) = b$ instead
  of $(a, b)\in f$
  \begin{itemize}
  \item the \emph{domain} of $f$ is
    $\dom{f} = \{a \mid \exists b \in B.\ f(a) = b \}$;

  \item we write $f(a) \downarrow$ for $a \in dom (f)$ and
    $f(a) \uparrow$ for $a \not\in dom (f)$;
  \end{itemize}

\item Given a set $A$ we indicate with $|A|$ its \emph{cardinality}
  (intuitively, the number of elements of $A$, but the notion extends
  to infinite sets). Given the sets $A$ and $B$ we have 
  \begin{itemize}
  \item $|A| = |B|$ if there exists a bijective function $f : A \to B$;
  \item $|A| \leq |B|$ if there exists an injective function
    $f: A \to B$ injective or equivalently\footnote{Stritly speaking,
      the equivalence requires the axiom of choice.} a surjective
    function $g : B \to A$.
  \end{itemize}
  Observe that if $A \subseteq B$ then $|A| \leq |B|$ as witnessed by
  the inclusion, which is an injective function
  \begin{quote}
    $\begin{array}{cc}
       i: & A \to B  \\
          & a \mapsto a
     \end{array}$
   \end{quote}
   
 \item We say that $A$ is \emph{countable} or \emph{denumerable} when
   $|A| \leq |\nat|$, i.e., there is a surjective function
   $f: B \to A$. Note that, when this is the case, we can
   list (enumerate, whence the name) the elements of $A$ as
   \begin{center}
     $\begin{array}{cccc}
        f(0) & f(1) & f(2) & \dots\\
        a_0  & a_1  & a_2 & \dots
      \end{array}
      $
    \end{center}

  \item When $A, B$ are countable then $A\times B$ is countable.
    
    Idea of the proof:
    \begin{itemize}
    \item Since $A$ and $B$ are countable, we can consider the
      corresponding enumerations
      
      \begin{quote}
        $
        \begin{array}{cccc}
          A & a_0 & a_1 & a_2 \\
          B & b_0 & b_1 & b_2
        \end{array}
        $
      \end{quote}
      and place the elements of $A \times B$  in a sort of matrix
      \begin{center}
        $
        \begin{tabu}{c|ccc}
          & b_0       & b_1       & b_2       \\
          \hline
          a_0 & (a_0,b_0) & (a_0,b_1) & (a_0,b_2) \\
          a_1 & (a_1,b_0) & (a_1,b_1) & (a_1,b_2) \\
          a_2 & (a_2,b_0) & (a_2,b_1) & (a_2,b_2)
        \end{tabu}
        $
      \end{center}
      in a way that they can be enumerated following along the diagonals
      as follows:
      $(a_0,b_0), (a_0,b_1), (a_1,b_0), (a_0,b_2), (a_1,b_1), (a_2,b_0),
      \dots$ (this is referred to as ``dove tail'' enumeration)
    \end{itemize}    


  \item A countable union of countable sets is countable: if
    $\{A_i\}_{i\in\nat}$ is a collection of countable sets then
    $\bigcup \limits_{i \in \nat} A_i$ is countable.
  \end{itemize}

  \section{Existence of non-computable functions in each computational model}

  Let us considered some fixed computational model satisfying the
  assumptions in \S\ref{se:alg-char}. We want to show that there are
  functions which are not computable in such a model.

  We focus on unary functions over the natural numbers. Let
  $\mathcal{F} = \{f \mid f:\nat\rightarrow\nat\}$ be the set of all the
  (partial) unary functions on $\nat$.

  Let $\mathcal{A}$ be the set of all algorithms in our fixed
  computational model.
  % 
  Every algorithm $A \in \mathcal{A}$ computes a function
  $f_A: \nat \to \nat$ and a function is computable in our model if
  there exists an algorithm that computes it. Hence the set
  $F_\mathcal{A}$ set of computable functions in the given computational
  model is
  \begin{center}
    $\mathcal{F}_{\mathcal{A}} = \{ f_A \mid A \in \mathcal{A} \}$.
  \end{center}

  Certainly $\mathcal{F}_A \subseteq \mathcal{F}$. But, is the inclusion
  is strict, i.e., is there a non-computable function?

  The answer is yes, essentially for combinatory reasons: algorthms are
  too few to compute all functions.


  In fact, an algorithm $A \in \mathcal{A}$ will be a finite, by
  assumption (\ref{as:prog_fin}), sequence of instructions taken from
  some instruction set $I$. Morever, by assumption (\ref{as:istr_fin}),
  $I$ must be finite. Hence
  \begin{center}
    $\mathcal{A} \subseteq \bigcup_{i \in \nat} I^n$
  \end{center}
  Since  a countable union of finite (hence countable) sets is countable, we have
  \begin{center}
    $|\mathcal{A}| \leq |\bigcup_{n\in\nat} I^n| \leq |\nat|$
  \end{center}
  and since the function
  \begin{quote}
    $\mathcal{A} \to F_\mathcal{A}$\\
    $A \mapsto f_A$
  \end{quote}
  is surjective by definition, we have that
  \begin{center}
    $|F_\mathcal{A}| \leq |\mathcal{A}| \leq |\nat|$
  \end{center}

  On the other hand the set of all functions, $\mathcal{F}$, is not countable. Let $\mathcal{T}$ the subset of $\mathcal{F}$ consisting of the total functions $\mathcal{T} = \{ f \mid f \in \mathcal{F}\ \land\ \dom{f} = \nat\}$. We show that
  \begin{center}
    $|\mathcal{F}| \geq |\mathcal{T}| > |\nat|$.
  \end{center}

  We prove that $|\mathcal{T}| > |\nat|$ by contradiction. Let us suppose that $\mathcal{T}$ is countable. Then we can consider an enumeration $f_0, f_1, f_2, \ldots$ of $\mathcal{F}$ and thus a matrix like the following
  \begin{center}
    \begin{tabular}{c|ccc}
      & $f_0$    & $f_1$    & $f_2$\\ 
      \hline
      0 & $f_0(0)$ & $f_1(0)$ & $f_2(0)$ \\
      1 & $f_1(0)$ & $f_1(1)$ & $f_1(2)$ \\
      2 & $f_2(0)$ & $f_2(1)$ & $f_2(2)$
    \end{tabular}
  \end{center}
  and build a function, that consists of the values on the diagonal, systematically changed:
  \begin{quote}
    $d: \nat \to \nat$\\  
    $d(n) = f_n(n)+1$
  \end{quote}

  We can observe that
  \begin{itemize}
  \item $d$ total, by definition;
  \item $d \neq f_n$ for all $n \in \nat$ (since $d(n) = f(n)+1 \neq f(n)$.
  \end{itemize}
  This is absurd, since $f_0, f_1, f_2, \ldots$ is an enumeration of all the total functions.

  \medskip

  Summing up:
  \begin{center}
    $\mathcal{F}_A \subsetneq F$ and
    $|F_A| \leq |\nat| < |\mathcal{T}| = |\mathcal{F}|$
  \end{center}
  therefore $F_A \subsetneq F$, as desired.

  Note that the non-computable functions are not countable:
  \begin{center}
    $|\mathcal{F} \setminus \mathcal{F}_{\mathcal{A}}| > |\nat|$.
  \end{center}
  In fact, $\mathcal{F} = \mathcal{F}_{\mathcal{A}} \cup (\mathcal{F} \setminus \mathcal{F}_{\mathcal{A}})$. Thus, if it were $|\mathcal{F} \setminus \mathcal{F}_{\mathcal{A}}| \leq |\nat|$, since the union of countable sets is countable, we would have $\|\mathcal{F}| \leq |\nat$.

  We conclude that
  \begin{enumerate}
  \item no computational model can compute all  functions;
  \item the non-computable functions are the majority.
  \end{enumerate}
