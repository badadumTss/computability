\chapter {Recursively enumerable set}

$ A \subseteq \nat $ if the semi-characteristic function is computable.
\begin{equation*}
  sc_A(x) = \begin{cases}
    1        & x \in A        \\
    \uparrow & $ altrimenti $
  \end{cases}
\end{equation*}

Predicate $ Q(\vec{x}) \subseteq \nat^k $ semi-decidable

\begin{equation*}
  sc_Q(\vec{x}) = \begin{cases}
    1        & Q(\vec{x})     \\
    \uparrow & $ altrimenti $
  \end{cases}
\end{equation*}

So say $A$ is r.e. is like saying that the predicate $ Q(x)=``x \in A" $ is semi-decidable

\textbf{Proprietà}: All recursive sets are also recursively enumerable:

\textit{A} recursive if $ A, \bar{A} $ r.e.

If $A$ recursive, \begin{equation*}
  \mathcal{X}_A(x)=\begin{cases}
    1 & x\in A        \\
    0 & $ otherwise $
  \end{cases}
\end{equation*}

Then $ sc_A(x) = \mu z.\bar{sg}(\mathcal{X}(x)) + 1 $ computable. Computable, therefore \textit{A} is r.e. Same goes for $ \bar{A} $.

\section {Theorem of structure of semidecidable predicates}

Let $ Q(\vec{x}) \subseteq \nat^k $ be a predicate.

This is decidable $ \Leftrightarrow $ there is a predicate $ Q'(t,\vec{x}) \subseteq \nat^{k+1} $ s.t. $ Q(\vec{x}) = \exists t. Q'(t,\vec{x}) $

So try 0,1,2, \dots if there is a point where it holds then yes, otherwise I try endlessly.

\section {Projection theorem}

Let $ P(x,\vec{y}) $ be semi-decidable; then $ \exists x $ s.t. $ P(x,\vec{y}) = P'(\vec{y})$ is semi-decidable.

So if you use existential identifier you go out of the set of the decidable predicates and enter the semi-decidable set, but if you use it twice you don't go outside the semi-decidable set.

\textbf{observation:} $ P_1(\vec{x}), P_2(\vec{x}) $ semi-dec. $ \Rightarrow P_1(\vec{x}) \lor P_2(\vec{x}) $; $ P_1(\vec{x}) \land P_2(\vec{x}) $ semi-dec.

Because you quantify existentially a single number whose components are equivalent to quantifying the two numbers of the individual predicates. In the \textit{OR} case and in the \textit{AND} case you look for the same number directly.

\textbf{Exercize:} If $ P(\vec{x}) $ is semi-decidable and is not decidable then $ \lnot P(\vec{x}) $ is not semi-decidable.

\textbf{Observation:} $ A,B \subseteq \nat, A\leq_m B $ then:
\begin{itemize}
\item B is r.e. $ \Rightarrow $ A is r.e .;
\item A is not r.e. $ \Rightarrow $ B not r.e.
\end{itemize}
Demonstration: If B r.e. then
\begin{equation*}
  SC_B(x) = \begin{cases}
    1        & x \in B       \\
    \uparrow & $ otherwise $
  \end{cases}
\end{equation*}
This is computable. Let $ f:\nat\rightarrow\nat $ be a total computable reduction function\\  $ A\leq B $ Then $ SC_A(x) = SC_B(f(x)) $, therefore $ SC_A $ is computable by composition.
