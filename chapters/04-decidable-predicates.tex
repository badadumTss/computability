\chapter{Decidable Predicates}

In mathematics we often want to establish \textbf{properties}, for example: $ div(m,n)  \quad  div \subseteq \nat \times \nat \quad  div = \{(m,Km) \mid m \in \nat, K \in \nat \} $

We can say $ div: \nat \times \nat \rightarrow \{0,1\} $|

In the field of calculability theory we talk about \textbf{predicates} or \textbf{problems}.

A \textbf{K-ary predicate} on $\nat $ indicated with $Q(x_1,x_k)$ is therefore a property that can be true or false, formally we can see it as:

\begin{itemize}
\item a function $Q: \nat^k\rightarrow \{true,false\}$
\item a set $Q \subseteq \nat^k$
\end{itemize}

we write $Q(x_1,\dots,x_k)$ to indicate $(x_1,\dots,x_k) \in Q$ or $Q(x_1,\dots,x_k) = true$

When is it computable? Int. when there exists a URM where given a k-tuple $(x_1,\dots,x_k)$ as input, it returns $true$ if $Q(x_1,\dots,x_k)$ and $false$ otherwise. To represent $true$ and $false$ we use conventionally use values 1 and 0.

\subsection{Definition of decidable predicate}

A predicate $Q \subseteq \nat^k$ is said to be \textbf{decidable} if its \textbf{characteristic function}

\begin{equation*}
\mathcal{X}_Q(x_1,\dots,x_k) = \begin{cases}
1 & $ if $ Q(x_1,\dots,x_k) \\
0 & $ otherwise $
\end{cases}
\end{equation*}

is computable (for now URM-computable).

\textbf{Note} $\mathcal{X}_Q$ is a \textbf{total} function (when one studies the decidability of predicates, one works only with total functions)

\section {Example of decidable predicate}

The example is equality. $ Q(x,y) \subseteq \nat^2 $, $ Q(x,y) \equiv x = y $

In fact the characteristic function

\begin{equation*}
\mathcal{X}_a(x,y) = \begin{cases}
1 & $ if $ x = y  \\
0 & $ otherwise $
\end{cases}
\end{equation*}

is computed by the program

\begin{quote}
\begin{tabular}{lll}
& J(1,2,SI)  \\
NO:  & Z(1)       \\
& J(1,2,END) \\
SI:  & Z(1)       \\
& S(1)       \\
END: &
\end{tabular}
\end{quote}

Another example: $ Q(x) \equiv x $ is even:

\begin{quote}
\begin{tabular}{lll}
LOOP: & T(1,2,SI)   \\
& S(2)        \\
& J(1,2,NO)   \\
& S(2)        \\
& J(1,2,LOOP) \\
SI:   & S(3)        \\
NO:   & T(3,1)
\end{tabular}
\end{quote}

\begin{tabu}{|c|c|c|}
\hline
x & k & r \\
\hline
\end{tabu} in memory where k is a growing index and r is the result.

Another example is $Q(x,y) \equiv x \leq y$. We can either increment both $x$ and $y$ until $x+k=y$, so that $x\leq y$, or until $y+k=x$, so that $x>y$ (not equal for the order of comparisons).

\begin{quote}
\begin{tabular}{lll}
& T(1,3)      &        \\
& T(2,4)      &        \\
LOOP: & J(2,3,SI)   & \comment{x+k=y?} \\
& J(1,4,NO)   & \comment{y+k=x?} \\
& S(3)        &        \\
& S(4)        &        \\
& J(1,1,LOOP) &        \\
SI:   & S(5)        &        \\
NO:   & T(5,1)      &
\end{tabular}
\end{quote}

Memory: $\begin{tabu}{|c|c|c|c|c|}
\hline
x & y & x+k & y+k & r \\
\hline
\end{tabu}$ where $r$ is the result.

Another approach is to increment a register starting from 0. If we reach $x$ first then $x \leq y$, otherwise $x > y$.

\begin{quote}
\begin{tabular}{lll}            
LOOP: & J(1,3,SI)   & \\
& J(2,3,NO)   & \\
& S(3)        & \\
& J(1,1,LOOP) & \\
SI:   & S(4)        & \\
NO:   & T(4,1)      &
\end{tabular}
\end{quote}

$\begin{tabu}{|c|c|c|c|}
\hline
x+k & y & k & r \\
\hline
\end{tabu}$ where $r$ is the result.

Example of $div(x,y)$, suppose $x \not= 0$:

\begin{quote}
\begin{tabular}{lll}            
LOOP: & J(2,3,SI)   &                                   \\
& Z(4)        & \comment{sum $x$ to $R_2$}         \\
ADDX: & J(1,4,LOOP) &                                   \\
& J(2,3,NO)   & \comment{if for $h<x$  $kx+h=y$ then no!} \\
& S(3)        &                                   \\
& S(4)        &                                   \\
& J(1,1,ADDX) &                                   \\
SI:   & S(5)        &                                   \\
NO:   & T(5,1)      &
\end{tabular}
\end{quote}

$\begin{tabu}{|c|c|c|c|c|}
\hline
x & y & kx+h & h & r \\
\hline
\end{tabu}$ where $r$ is the result.
