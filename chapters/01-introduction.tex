\chapter{Introduction}

In this chapter, we informally discuss the notion of \textbf{effective procedure} and \textbf{function computable} by the means of an effective
procedure. This will lead us to single out the main features of an
algorithm/computational model.  Despite being informal, these
considerations will allow us to derive the existence of non-computable
functions for any chosen effective computational model.
%
Later these notions and considerations are going to be formalised by fixing a
specific computational model, a sort of idealized computer.

\section{Algorithm or effective procedure (Informal)}

Even though we do not always refer to them by their technical terms when we apply them, \textbf{effective procedure}s and \textbf{algorithms} are a part of our everyday life.

For example; at the primary school we are not only taught that given two numbers their sum exists, but we are also provided a procedure to compute the sum of two numbers!

In general terms, an \emph{algorithm} can be defined as the
description of a sequence of \emph{elementary steps} (where
``elementary'' means that they can be performed mechanically, without
any intelligence) which allows one reach some objective.  Typically,
the aim is transforming some input into a corresponding output,
suitably related to the input.
%
This could be transforming ingredients into a cake, although normally
we are interested in computational problems.

\begin{example}
  Some examples are:
  \begin{enumerate}

  \item given $n \in \nat$ establish whether $n$ is prime;
  \item find the $n^{th}$ prime number;
  \item derive a polynomial;
  \item perform the square root $\sqrt{n}$;
  \item least common multiple \emph{lcm} and largest common divisor \emph{LCD}.
  \end{enumerate}
\end{example}

Therefore we can think of an algorithm as a black box.
\begin{center}
  in $\rightarrow$ \boxed{black box} $\rightarrow$ out
\end{center}
where the transformation is performed by executing a sequence of
``mechanical'' instructions.

If each step is \emph{deterministic} (i.e., in each state of the
system the instruction to execute and the new state it produces are
uniquely determined), then each possible input will uniquely determine
the corresponding output (the procedure might not terminate, in which case we will have no output).

In mathematical terms the algorithm induces a \emph{(partial) function}
\begin{center}
  $f : \mathit{input} \rightarrow \mathit{output}$.
\end{center}
We say that $f$ is the \emph{function computed} by the algorithm and
that $f$ is effectively computable. Thus, we can give the following
first definition of an algorithm (still informal since it refers to a
generic notion of algorithm).

\begin{definition}[computable function]
  A function $f$ is computable if \emph{there exists} an algorithm
  that computes $f$.
\end{definition}

We stress that for $f$ to be computable, it is not important to know the algorithm that computes $f$, but rather we need to know that some algorithm that computes $f$ exists.

\begin{example}
  According to the above definition, we expect the the following
  functions to be computable.

  \begin{itemize}

  \item GCD (greatest common divisor), e.g., exploiting Euclid's
    algorithm.

  \item the function $f : \nat \to \nat$ defined as

    \begin{equation*}
      f(n)=
      \begin{cases}
        1 & n \mbox{ is prime} \\
        0 &   \mbox{otherwise}
      \end{cases}
    \end{equation*}

  \item
    $g(n)= n$-th prime number\\
    (this is computable, maybe inefficiently by generating numbers
    and testing for primality until the $n$-th prime is found)


  \item
    $h(n) = n$-th digit of the decimal representation of $\pi$.

    In fact, from analysis we know that
    \begin{itemize}
    \item There are series that converge to $\pi$
    \item There are techniques to estimate (by excess) the error caused:
      \begin{itemize}
      \item by truncating a series
      \item by rounding in the calculation of the value of the truncated series
      \end{itemize}
    \end{itemize}
  \end{itemize}
\end{example}

What about the function below?

\begin{equation*}
  g(n) = \begin{cases}
    1 & $ there is a sequence of exactly  \textit{n} consecutive $5$'s in $ \pi \\
    0 & $ otherwise $
	\end{cases}
\end{equation*}

For example $g(3) = 1$ if and only if  $\pi = 3.14 \dots k 555 h \dots $, with $k, h \neq 5$.

Computability is unclear. A naive algorithm could be:
\begin{itemize}
\item generate the digits of $\pi$
\item until a sequence of $5$'s of the desired length $n$ is found.
\end{itemize}
Clearly, if such a sequence exists, it will be eventually found and the answer $1$ will be given. However, at no point the computation, not having found the desired sequence of $n$ 5's, we can certainly say that it will never appear later on! Hence, we currently have no way of answering $0$.

\begin{remark}
  Clearly, something of the kind:
   \begin{itemize}
   \item generate all digits in the decimal representation of $\pi$;
   \item if they include a sequence of $n$ consecutive 5's
    $\rightarrow g(n) = 1$
  \item otherwise $\rightarrow g(n) = 0$
\end{itemize}
is \emph{not} an effective procedure.
\end{remark}

Note that this doesn't mean that $g$ is not computable, i.e., that an
effective procedure couldn't be found (e.g. on the basis of properties
of $\pi$), but at the moment this procedure is not known! (at least by
me!)

We don't really know if it's computable, but there might be a property
of $\pi$ that allows us to conclude. In particular, there is a conjecture that all
finite sequences of digits appear in $\pi$, which would imply that $g$
is the constant $1$, whence computable.

\medskip

Consider now a slightly different function $h: \nat \to \nat$, defined by
\begin{equation*}
  g(n) = \begin{cases}
    1 & $ there is a sequence of at least  \textit{n} consecutive $5$'s in $ \pi \\
    0 & $ otherwise $
  \end{cases}
\end{equation*}

The function seems very similar to the one considered before. However, note that if $\pi = 3.14 \dots k 555 h \dots $, then we deduce, not only that $h(3)=5$, but also $h(2)=h(1)=h(0)=1$. More generally, whenever $h(n) =1$ then $h(m)=1$ for all $m < n$. This suggests that $h$ could be quite ``simple''.

More precisely, consider $K = sup\{ n \mid \pi\ \text{contains}\ n\ \text{consecutive digits}\ 5 \}$. Then we have 2 possibilities:
\begin{enumerate}
\item $K$ is finite, and thus $h(n) = 1$ if $ n\leq k$, $0$ otherwise
\item $K$ is infinite, and thus $ h(n) = 1$ for all $n \in \nat$
\end{enumerate}

This implies that $h$ is computable because it is either a step
function or a constant function, that are computed by simple
programs. One could object that we don't know which shape the function
has and thus we do not know the program that calculates the
function. Fine. This doesn't mean that it's not computable.

Trying to repeat the same argument for function $g$ fails. In fact, one could think of defining $A = \{n \mid \mbox{$\pi$ contains exactly $n$ consecutive 5's}\}$. Then

\begin{equation*}
  g(x) =
    \begin{cases}
      0 & x \in A     \\
      1 & x \not\in A
    \end{cases}
\end{equation*}

This does not suggest that $g$ is computable. Set $A$ is possibly infinite and we do not see a way of providing a finite representation of $A$ which can be included in a program.

Bringing the argument to the extreme, one could consider the function
$G : \nat \to \nat$ defined by
\begin{center}
  $G(x) = \begin{cases}
    1 & $ if God exists $ \\
    0 & $ otherwise $
  \end{cases}.
  $
\end{center}
This is either the constant $0$ or the constant $1$. Independently of which of the two cases applies, the function is computable.


TO BE PROCESSED

