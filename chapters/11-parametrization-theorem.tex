\chapter {Parameterization theorem}

Suppose we have a computable function $ f:\nat^2\rightarrow\nat$ therefore $ \exists e \in \nat . f = \phi_e^{(2)} $, now consider for the first fixed argument: $ f(x,y) = f_x(y) $, where $ f_x:\nat\rightarrow\nat $ computable, therefore there is a program that calculates it.

For example let's take $ f(x,y) = y^x $ then $ f_0(y) = 1 $, \dots all these functions are computable.

We had the program that computed $ f(x,y) = \phi_e^{(x)}(x,y) $, so there is some program $d$ that calculates $ f_x = \phi_d $, but obviously they are all different programs for each $d$. We observe that this program depends on \textit{e} and on \textit{x}.

What we are saying is that there exists total $ s:\nat^2\rightarrow\nat$ such that $ \phi_{s(e,x)}(y) = \phi_e^{(2)}(x,y)$, the theorem says it is computable. In general we can take a function of $ m+n $ arguments, $ m,n \in \nat $, $ m,n \geq 1 $ there exists $ S_{m,n} : \nat^{m+1}\rightarrow\nat \in \mathcal{PR}$ s.t. \begin{equation*}
  \phi_e^{m+n}(\vec{x},\vec{y}) = \phi_{s_{m,n}(e, \vec{x})}^{n}(\vec{y})
\end{equation*}

But now let's see how to calculate $ \gamma $. First we define function $ agg:\nat^2\rightarrow\nat $ where $ agg(e,t) =$ program obtained by \textit{e} adding \textit{t} to the destination of each jump.

Now let's take $ \overline{agg} $ where $ \overline{agg}(i,t) = $ update of instruction i (meaning $ \beta^{-1}(i) $)

\dots

\section {Corollary SMN theorem}
Given the function $ f:\nat^{n+m}\rightarrow\nat,\\ \exists S:\nat^m\rightarrow\nat $ total computable s.t. $ f(\vec{x},\vec{y}) = \phi_{s(\vec{x})}^{(n)}(\vec{y}) \forall \vec{x},\vec{y}$

\section {Exercise}

Show that there exists $s$ total computable s.t. $ \phi_{s(n)}(x) = \sqrt[n]{x} $

\textbf{Execution:} I take the function $ f(n,x) = \sqrt[n]{x} $ so I have to search \\ $ max(y) . y^n \leq x $, but I can only use the minimum, so I try $ min(y). (y+1)^n > x $

that is: $ \mu y \leq x . x+1 - (y+1)^n $. Even without bound it worked, but so we also show that it is recursive primitive (it was not required).

We see that it is computable, therefore by corollary of the smn theorem there exists the total computable function $s$ such that $ f(n,x) = \phi_{s(n)}(x) $.
