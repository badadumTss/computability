% places to put matrices to build: ``complex matrix''
\chapter{Cantor diagonalization method}

The diagonalization technique, in a general sense, allows to build an
object that differs from an enumerable infinity of similar objects.
The idea behind it is that when given an enumerable et of objects
(functions, numbers, ...) with structure $\{x_1, x_2, x_3, \dots \}$
we can build another object $x$ based on the structure itself with the
same nature of all the other objects, but different from each of them,
by making it ``differ from $x_n$ on $n$''.

\begin{example}[$|2^\nat| > |\nat|$]
  This is the original use of the Cantor method, from Cantor himself,
  founder of the modern theory of sets, to prove that there are
  various ``levels of infinity''.
  \begin{proof}
    Lets suppose that $|2^\nat| = |\nat|$. This means that exists an
    enumeration of $2^\nat$: $x_0, x_1, x_3, \dots$

    Consider
    % Complex to draw matrix

    obviously $X \in 2^\nat \rightarrow \exists k$ s.t. $X = X_k$. But
    is $k \in X$?

    \(k \in X \quad \Rightarrow \quad k \notin X_k = X\)
    
    \(k \notin X \quad \Rightarrow \quad k \in X_k = X\)

    Which is absurd. Therefore $2^\nat$ is not enumerable.
  \end{proof}
\end{example}

\newcommand{\nattonat}{\nat \rightarrow \nat}
\begin{corollary}\label{corollary:nattonat}
  if $\nattonat = \{$ partial functions from $\nat$ to $\nat \}$
  then \[|\nattonat| > |\nat|\]

  \begin{proof}
    We can observe that
    \[2^\nat = |\{f: \nat \rightarrow \{0,1\} \; | \; f \mbox{ total }
      \} | \; \leq | \nattonat | \]

    and so \[|\nattonat| \geq |2^\nat| > |\nat|\]

    Alternatively we can prove it with the diagonalization
    technique. Let $f_1, f_2, f_3,$ be an enumeration of elements in
    $\nattonat$ and consider

    % again, complex matrice of elements

    we can define a functoin $f$ that differs from every other
    function on the diagonal based on the input:
    \[f(n) = \begin{cases}
        0 & \quad \mbox{if } f_n(n)\uparrow \\
        \uparrow & \quad \mbox{if } f_n(n) \downarrow
      \end{cases}
    \] \[f \in \nattonat\] so that
    $\forall n \quad f(n) \neq f_n(n) \quad \Rightarrow f \neq f_n$
  \end{proof}
\end{corollary}

\newcommand{\noc}{\bar{\mathcal{C}}}

\textbf{Note:} the set
$\noc = \{f : \nattonat \; | \; f \mbox{ not computable}\}$ is not
enumerable.

\begin{proof}
  We know that $|\mathcal{C}| = |\nat|$. if $\noc$ was enumerable then
  $\nattonat = \mathcal{C} \cup \noc$ would be enumerable, which is
  absurd for the corollary \ref{corollary:nattonat}.
\end{proof}

\begin{exercise}
  Exists a total function non computable. We lready knew that, but for
  cardinality reasons, now we are able to exibhit it.
  \[
    f(n) = \begin{cases}

      \varphi_n(n) + 1 & \quad \mbox{if } \varphi_n(n) \downarrow (n
      \in W_n) \\

       0 & \quad \mbox{if } \varphi_n(n)\uparrow (n \notin W_n)
    \end{cases}
  \]

  again, with the Cantor method we can build
  \[f(0) = \varphi_0(0), f(1) = \varphi_1(1), f(2) = \varphi_2(2),
    \dots \]
  is easy to see that
  \begin{itemize}
  \item $f$ is total
  \item $\forall n \; f \neq \varphi_n \; \Rightarrow f$ is not
    computable
  \end{itemize}
\end{exercise}

\textbf{Note:} there are infinitely many total non computable
functions in the form

\[
  f(n)  = \begin{cases}
    \varphi_n(n) + k & n \in W_n \\
    k & n \notin W_n
  \end{cases}
\]

% the prof here wrote "how many are them?", but....

\begin{exercise}
  Let $f: \nat \rightarrow \nat$ (partial), $m \in \nat$

  Define a function $g : \nat \rightarrow \nat$ non computable
  s.t. \[g(x) = f(x) \quad \forall x < m\]

  You can use a ``translated diagonal'';

  % again, complex matrix

  \[
    g(x) = \begin{cases}
      f(x) & x < m \\
      \varphi_n(x) + 1 & x \geq m \mbox{ and } x \in W_{x-m} \\
      0 & x \geq m \mbox{ and } x \notin W_{x-m}
    \end{cases}
  \]

  clearly $\forall x \quad g\neq \varphi_x$ since
  $g(x + m) \neq \varphi_x(x+m)$.

  \textbf{Note:} we could have defined

  \[
    g(x) = \begin{cases}
      f(x) & x < m \\
      \varphi_n(x) + 1 & x \geq m \mbox{ and } x \in W_{x} \\
      0 & x \geq m \mbox{ and } x \notin W_{x}
    \end{cases}
  \]

  why? Each function appears infinitely many times in the enumeration,
  and skip the first $m-1$ steps doesn't create any
  problem... Formally we know that
  $\forall x \geq m \quad g \neq \varphi_x$ so $\forall y$ since there
  are infinitely many indices for $\varphi_x$
  \[\exists x \geq m \mbox{ s.t. } \quad \varphi_y = \varphi_x \quad
    \mbox{and } \varphi_x \neq y\] so, $\forall y \; \varphi_y \neq g$
  (g is not computable).
\end{exercise}

\begin{exercise}
  Given a family of functions
  $\{f_i\}_{i\in \nat} \quad f_i : \nattonat$ define $g: \nattonat$
  s.t. $\dom{g} \neq \dom{f_i} \quad \forall i \in N$

  \[
    g(n) = \begin{cases}
      0 & \mbox{if } n \notin \dom{f_n} \\
      \uparrow & \mbox{if } n \in \dom{f_n}
    \end{cases}
  \]

  This way $\forall n \quad n \in \dom{g}$ if and only if
  $n \notin \dom{f_n}$.
\end{exercise}

\begin{exercise}
  Deifne a non computable total function that returns 1 when the input
  is even
  \[
    f(x) = \begin{cases}
      1 & x \mbox{ is even} \\
      \varphi_{\frac{x-1}{2}} + 1 & x \mbox{ is odd, and } x \in
      W_{\frac{x-1}{2}} \\
      0 & x \mbox{ is odd, and } x \notin W_{\frac{x-1}{2}}
    \end{cases}
  \]

  is easy to prove that the function $f$ is not computable. In fact,
  $\forall n \quad f(2n + 1) \neq \varphi_n(2n+1)$. This means that

  if
  $2n+1 \in W_n \Rightarrow f(2n+1) = \varphi(2n+1) + 1 \neq
  \varphi_n(2n+1)$

  if
  $2n+1 \notin W_n \Rightarrow f(2n+1) = 0 \neq \varphi_n(2n+1)
  \uparrow $

\end{exercise}
% Same as the previous chapter, these notes are very different form
% the italian version and not complete in the same way

% Given a set $X$ we know that $ |X| \geq |2^X| $ is never valid, that
% is, the set of its parts is always bigger. Suppose there is
% $ f:X\rightarrow2^X $ surjective. Hence $ R = \{y . y \in X $ and
% $ x \not \in f(y) \ \} \in 2^X $, since f is surjective then
% $ \exists y_R \in X . f(y_R) = R$. I consider the cases separately:
% \begin{itemize}
% \item $ y_R \in R $ then $f(y_R) \Rightarrow y_R \not \in R $
% \item $ y_R \not \in R $ then $ f(y_R) \Rightarrow y_R \in R$
% \end{itemize}
% Now we prove that the set of the parts of the natural numbers is not
% countable. We assume there is a surjective function
% $ \nat \rightarrow 2^\nat $. This means that I can enumerate subsets
% $ X_i $ s.t. $ i \in \nat $. At this point I create a matrix where
% each row $i$ of column $j = 1$ if $ x_i \in X_j $ and 0 otherwise. Now
% consider the inverted diagonal, that is if $ x_i \not \in X_i $, in
% this way it is different from all the columns, that is
% $ \forall i . R \not= X_i $ because
% $ n \in R \Leftrightarrow b \not \in X_n $ absurd since I assumed that
% $ \{X_0 \dots X_n \} = 2^\nat$.

% We conclude that $ |\nat| \not \geq |2^\nat| $

% But we want $ |\nat| < |2^\nat| $.

% But:
% $ |\nat| \leq |2^\nat| \land |\nat| \not\geq |2^\nat| \implies |\nat|
% < |2^\nat| $

% I take the set of the characteristic functions $g$ of $ 2^\nat $ and I
% call it $Y$, I take the set of all the functions $ \mathcal{F} $, of
% course it holds that $ Y \subseteq \mathcal{F} $ and therefore
% $ |Y| \leq |\mathcal{F}| $ but being that $ |\nat < |2^\nat| $ then I
% also have that $ |\nat| \leq |\mathcal{F}| $

% There is a total function that cannot be computed. We know how to
% enumerate computable functions because we can enumerate them in the
% form of numbers, repeating some of them. Now let's define $ f(n) $ as
% non-total computable.

% Matrix where the columns are the functions computed by the program
% $ i \in \nat $, that is $ \{\phi_i . i \in \nat \} $ and the rows are
% the arguments 0,1, \dots.

% \begin{equation*}
%   f(n) = \begin{cases}
%     \phi_n(n)+1 & $ se $\phi(n)\downarrow \\
%     0           & $ se $\phi(n)\uparrow
%   \end{cases}
% \end{equation*}

% we observe that $f$ is total by construction and
% $ f \not= \phi_n \forall n, n \in \nat $ Furthermore by construction
% it is different from all computable functions and therefore it is not
% computable.
