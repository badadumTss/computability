\chapter {Rice theorem}

Rice thorem gives an undecidability result very general. In broad
terms it states that \emph{no property} of computable functions
(besides obvois ones) are decidable.

To state it in terms of of subsets of $\nat$, we need the following
notion

\section{Saturated sets}
\begin{definition}[Saturated set]
  A subset $A \subseteq \nat$ is \emph{saturated} if
  \[
    \forall x \in A \wedge \varphi_x = \varphi_y \Rightarrow y \in A \}
  \]
\end{definition}

In other words $A$ is saturated if it expresses a property of
functions, independently from indices
\[A = \{x \mid P(\varphi_x)\}\]
or, again, if exists $\mathcal{A} \subseteq \mathcal{C}$ s.t.
\[A = \{ x \mid \varphi_x \in \mathcal{A}\]

\begin{example}
  \[
    T_2 = \{ e | P_e(e)\downarrow \mbox{ in two steps } \} =
    \{e|\phi_e \in \mathcal{T}_2 \}
  \]

  But two programs can calculate the same function and finish one in
  less than 2 steps and the other in more than 2, so the set is not
  saturated.
\end{example}

\begin{example}
  
  \begin{center}
    $ K = \{e \mid e\in W_e \} = \{e \mid \phi_e\in \mathcal{K} \}
  \mathcal{K} = \{f \mid ? \}$ I don't know what to write in this.
  \end{center}

  It is not actually saturated. Difficult to prove this, but we can
  show that there is a program $e$ such that:
  \begin{equation*}
    \phi_e(x) = \begin{cases}
      0 & x = e \\
      \uparrow & $ else $
    \end{cases}
  \end{equation*}
  And if you try to write such a program you realize that it is
  impossible.
\end{example}

\section {Rice theorem}

\begin{theorem}[Rice theorem]
  Let $ A \in \nat $ be saturated. $ A \neq \emptyset, A \neq \nat $
  then it is non-recursive.
\end{theorem}

\begin{proof}
  We wanto to prove that $ K \leq A $. We will show that it reduces,
  that is, we'll find $f$ total and computable s.t. all the elements
  of $K$ go to $A$ and all the elements of the complenent of $K$ go to
  the complement of $A$.

  \begin{itemize}
  \item[($e_0 \notin A$)]
    Let $ e_0$ s.t. $ \phi_{e_0}(x)\uparrow\forall x $ suppose
    $ e_0\notin A $ and let $ e_1\in A $
    
    Now let's define the following function:
    \begin{equation*}
      g(x,y) = \begin{cases}
        \phi_{e1}(y) & x \in K \\
        \phi_{e0}(y) & x \notin K
      \end{cases}
    \end{equation*}

    Which is equal to:

    \begin{equation*}
      g(x,y) =
      \begin{cases}
        \phi_{e1}(y) & x \in K \\
        \uparrow & x \notin K
      \end{cases}
    \end{equation*}

    and to calculate if $x$ is in $K$, we just need to run $ \phi_e(x) $
    and see if it ends.

    For the \smn theorem there exists $f$ s.t.
    $ \phi_{f(x)} = \phi_{e1} \Rightarrow f(x)\in A$

    \begin{equation}\label{eq:one14}
      x \in K
      \Rightarrow x \in W_x
      \Rightarrow \varphi_{f(x)}(y) = g(x,y) = \varphi_{e_1}(x)
      \Rightarrow e_1 \in A
    \end{equation}
    \begin{equation}\label{eq:two14}
      x \notin K
      \Rightarrow x \notin W_x
      \Rightarrow \varphi_{f(x)}(y) = g(x,y)\uparrow \forall y
      \Rightarrow e_0 \notin A
    \end{equation}

    And since $A$ is saturated, for (\ref{eq:one14})
    $x \in K \Rightarrow S(x) \in A$, for (\ref{eq:two14})
    $x\notin K \Rightarrow S(x) \notin A$.

    This way we proved that $K \leq_m A$, and since $K$ is not
    recursive, also $A$ is not recursive.
    
  \item[($e_0 \in A$)] if $ e_0 \in A $ then $ e_0 \not \in \bar{A}
    $. But we remember that
    $ \bar{A} \subseteq \nat, \bar{A} \neq \emptyset, \bar{A} \neq
    \nat $ then for the last point $ \bar{A} $ non-recursive and
    therefore neither $A$ is recursive.
  \end{itemize}
\end{proof}

\begin{example}
  \[ T = \{e \mid \varphi_e \mbox{ total } \} \mbox{ non-recursive} \]

  $T$ is saturated: $ T = \{e \mid \varphi_e \in \mathcal{T} \} $, $
  \mathcal{T} = \{f \mid f $ total $ \} $, $ T\neq\emptyset $, $ e_1 $\\
  s.t. $ \varphi_{e_1}(x) = 0 \quad \forall x $

  $ e_0 \notin \mathcal{T} $. Saturated, other than empty and $ \nat $
  therefore for Rice theorem it is non-recursive.
\end{example}

\begin{example}[output problem]
  \[ B_n = \{e \mid n \in E_e \} \quad \mbox{is not saturated} \]

  In this case we can use Rice theorem, since
  \begin{itemize}
  \item $B_n$ is saturated;
  \item $B_n \neq \emptyset$
  \item $B_n \neq \nat$ 
  \end{itemize}

  Hence $ B_n $ non-recursive.
\end{example}