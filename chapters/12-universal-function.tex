\chapter {Universal Function}
\newcommand{\Psiex}{\ensuremath{\Psi_{\mathcal{U}}^{(k)} (e, \vec{x})}}
\newcommand{\Psiuex}{\ensuremath{\Psi_e^{(k)} (\vec{x})}}
\newcommand{\univ}{\ensuremath{\Psi_{\mathcal{U}}}}
We'll now see how the theory we described up until now allows us to
prove something a bit surprising. This is to say, the existence of
universal functions/programs, able to reproduce the behaviour of each
and every other computable function/program. Let's consider the function

\[
  \Psi(x,y) = \varphi_x(y)
\]

We can already observe that $\Psi$ is a ``universal function'', that
captures all unary functions $\varphi_1, \varphi_2, \dots$. In fact,
for all fixed $e \in \nat$
\[
  g(y) = \Psi(e,y) = \varphi_e(y) \quad \rightsquigarrow \quad g = \varphi_e
\]
so, thanks to $e$, $\Psi$ rapresents all computable functions in the
form $\nattonat$.

\begin{definition}
  The universal function for k-ary functions (with $k \in \nat$) is
  defined as
  \[
    \univ^{(k)} : \nat^{k+1} \rightarrow \nat
  \]
  \[
    \Psiex = \Psiuex
  \]
\end{definition}
$\Psi$ functions are this well defined, but are they computable? If
the answer is yes a program $P_{\mathcal{U}}$ computing $\Psi$ would
be able to compute all k-ary functions. It includes in itself all
other computable programs, and we can call it a Universal Computer
\cite{davis:2011}.

At first glance this might seem odd, but tinking again it recives in
input:
\begin{itemize}
\item $e$ (the index of the program, a \textit{description} of the
  program $P_e$ to run)
\item $\vec{x}$ the arguments
\end{itemize}
We observe that it fits the description of an interpreter.

In fact, the following result applies

\begin{theorem}
  $\forall k \geq 1$ the universal function $\Psi_{\mathcal{U}}^{(k)}$
  is computable.

  \begin{proof}
    Informally, we can say that if we fix $k \geq 1$ and an index
    $e \in \nat$ and the arguments $\vec{x} \in \nat^k$ we can compute
    $\Psiex = \Psiuex$ as follows:
    \begin{itemize}
    \item we build the program $P_e = \gamma^{-1}(e)$;
    \item we simulate $P_e$ on input $\vec{x}$
    \item if $P_e(\vec{x})\downarrow$, the value of $\Psiex$ is in
      $R_1$, otherwise it is ok.
    \end{itemize}
    This way everything is effective, and for the Church-Turing
    thesis computable.

    But we want to be more formal than this. More precisely we'll need
    some programs in order to accomplish this:

    \paragraph{Configuration of registers}
    Given the registers of the machine $(r_1, r_2, \dots, r_n)$
    \[
      \begin{tabu}{|c|c|c|c|c}
        \hline
        r_1 & r_2 & r_3 & 0 & \dots \\ \hline
      \end{tabu}
    \]
    the \textit{configuration of registers} is given by
    \[ c = \prod_{i \geq 1} p_i^{r_i} \]
    This way the value of the registers $\forall i \quad r_i = (c)_i$

    \paragraph{The state of the machine}
    The state of the machine is encoded with \[ \sigma = \Pi(j, c) \]
    where $j$ is the next instruction to execute and $c$ is the
    configuration of registers.

    We want to show that the functions

    \[
      c_k : \nat^{k+2} \rightarrow \nat
    \]

    \[
      c_k(e, \vec{x}, t) = \begin{cases}
        % Insert the function definition we saw in class
        x & 0 \\
        y & 1
      \end{cases}
    \]

    \[
      j_k : \nat^{k+2} \rightarrow \nat
    \]

    \[
      j_k(e, \vec{x}, t) = \begin{cases}
        % Insert the function definition we saw in class
        x & 0 \\
        y & 1
      \end{cases}
    \]

    At this point we have that
    \[\Psiex = \Psiuex = (c_k(e,\vec{x}, \mu t \; . \; j_k(e, \vec{x}, t)))_1\]
    so if we prove that $c_k, j_k$ are computable we can conclude that
    also $\Psi_{\mathcal{U}}^{(k)}$ si computable.  We proceed in the
    same manner way we did in the proof \S\ref{reqc}, by proving that
    $c_k, j_k \in \pr$ (in fact, this can be seen as a more formal
    prove of the same fact, the only difference is that in the latter
    demonstration we defined $c_p, j_p$ with \emph{a fixed $P$}, here
    instead $P$ is a parameter). 

    Explictly, each step:

    \newcommand{\uarg}{{\mbox{arg}}}
    \newcommand{\uargh}{{\mbox{arg}_h}}
    \begin{enumerate}[label=(\alph*)]
    \item arguments of an URM instruction $( i = \beta(\mbox{Instruction}))$

      $Z_\uarg (i) = qt(4, i) + 1$
              
      $S_\uarg (i) = qt(4,i) + 1$
              
      $T_\uargh(i) = \Pi_h(qt(4,i)) + 1 \quad h \in \{1,2\}$
              
      $J_\uargh(i) = \nu_h(qt(4,i)) + 1 \quad h \in \{1,2,3\}$
      
    \item effect of executing an instruction on the configuration $C$

      \[
        \begin{tabu}{l l}
          \mbox{zero}(c,n) = qt(p_n^{(c)_n}, c) & Z(n) \\
          \mbox{succ}(c,n) = p_n \cdot c & S(n) \\
          \mbox{tfr}(c,m,n) = qt(p_n^{(c)_n}, x) \cdot p_n^{(c)_m} & T(m,n)
        \end{tabu}
      \]

    \item effect on the configuration of registers of the execution of
      the instruction $i=\beta(\mbox{Instruction})$

      \newcommand{\change}{\mbox{change}}
      \[
        \change(c,i) = \begin{cases}
          \mbox{zero}(c, Z_\uarg(i)) & rm(4,i) = 0 \\
          \mbox{succ}(c, S_\uarg(i)) & rm(4,i) = 1 \\
          \mbox{tfr}(c, T_{\uarg_1}(i), T_{\uarg_2}(i)) & rm(4,i) = 2 \\
          c & rm(4,i) = 3
      \end{cases} 
    \]

  \item configuration of the registers if the current one ($c$) and
    the $t$ instruction of $P_e$ is executed

    \newcommand{\nextconf}{\mbox{nextconf}}
    \[
      \nextconf(e,c,t) = \begin{cases}
        \change(c, a(e,t)) & 1 \leq t \leq \ell(e) \\
        c & \mbox{otherwise}
      \end{cases}
    \]

  \item next configuration of registersif the $t^{\mbox{th}}$
    instruction $i=\beta(\mbox{Instruction})$ is executed

    \newcommand{\instr}{\mbox{instr}}
    \[
      \instr(c, i, t) = \begin{cases}

        t+1 & (rm(4,1) \neq 3) \vee (rm(4,i) = 3 \wedge (c)_{J_{\uarg_1(c)}} \neq (c)_{J_{\uarg_2(i)}}) \\
        J_{\uarg_3(c)} & \mbox{otherwise}
      \end{cases}
    \]

  \item number of the next instruction if the $t^{\mbox{th}}$
    instruction of $P_e$ is execeuted on the configuration $c$

    \newcommand{\nextinstr}{\mbox{nextinstr}}
    \[
      \nextinstr(e,c,t) = \begin{cases}
        \instr(c, a(e,t), t) & 1 \leq t \leq \ell(e) \wedge \instr(c, a(e,t), t) \leq \ell(e) \\
        0 & \mbox{otherwise}
      \end{cases}
    \]
  \end{enumerate}

  At this point we can define $c_k$ and $j_k$

  $c_k(e, \vec{x}, 0) = \prod_{i=1}^kp_i^{x_i}$

  $j_k(e, \vec{x}, 0) = 1$

  $c_k(e, \vec{x}, t+1) = \mbox{nextconf} (e, c_k(e, \vec{x}, t), j_k(e,\vec{x},t))$

  $j_k(e, \vec{x}, t+1) = \mbox{nextinstr} (e, c_k(e, \vec{x}, t), j_k(e,\vec{x},t))$

  Therefore

  \[
    \sigma_k(e,\vec{x},t) = \Pi(j_k(e,\vec{x},t), c_k(e,\vec{x},t))
  \]

  can be defined with primitive recursion, therefore $\sigma \in \pr \Rightarrow c_k, j_j \in \pr$

  \[
    \Psiex = c_k(e, \vec{x}, \mu t \; . \; j_k(e,\vec{x},t)) \quad \in \mathcal{R} = \mathcal{C}
  \]
  \end{proof}
\end{theorem}

As a corollary, we obtain the decidability of two statements that will
be really useful in the next chapters.

\begin{corollary}
  The following predicates are decidable:
  \begin{enumerate}[label=(\alph*)]
  \item $H_k(e, \vec{x}, t) \equiv$ ``$P_e(\vec{x})\downarrow$ in $t$
    or less steps''
  \item $S_k(e, \vec{x}, y, t) \equiv$ ``$P_e(\vec{x})\downarrow y$ in
    $t$ or less steps''
  \end{enumerate}
  \begin{proof}
    \begin{enumerate}[label=(\alph*)]
    \item We observe that
      $H_k(e, \vec{x}, t) \equiv (j_k(e,\vec{x}, t) = 0)$ and therefore
      we have
      
      $\chi_{H_k}(e, \vec{x}, t) = \overline{sg}(j_k(e,\vec{x},t))$
    \item We observe that
      $S_k(e, \vec{x}, y, t) \equiv ((j_k(e,\vec{x},t) = 0) \wedge ((c_k(e,\vec{x},t))_1)=y)$
    \end{enumerate}
  \end{proof}
\end{corollary}

Also, from the theorem we deduce the possibility to express every
computable function in the so said Kleene normal form (KNF)

\begin{corollary}[Kleene Normal Form]
  $\forall e,k \in \nat \quad \forall x \in \nat^k$
  \[
    \Psiuex = (\mu z \;.\; |\chi_{S_k}(e, \vec{x}, (z)_1, (z)_2) - 1|)_1
  \]
\end{corollary}

\textbf{Observations:}
\begin{enumerate}[label=\roman*.]
\item This corollary highlights how each computable function (or
  equivlently $\in\pr$) can be obtained from primitive recursion
  functions using minimalization maximum one time (we can use just one
  \texttt{while})
\item Minimalization allows us to ``search'' a single value that has a
  certain property. The one we used is a technique to search couples
  of values generalizable to tuples.
\end{enumerate}

\section{Applications of the universal function}
Reminding that we already observed that if $f : \nattonat$ is a total
computable injective function, then
\[
  f^{-1}(x) = \begin{cases}
    y & \quad \mbox{if exists $y$ s.t. } f(y) = x \\
    \uparrow & \quad \mbox{otherwise}
\end{cases}
\]
is computable since $f^{-1} = \mu y \; . \; |f(y) = x|$. We can verify
that the hypotesis of \emph{totalyty} can be omitted.

\begin{theorem}
  Let $f: \nattonat$ computable and injective. Then
  $f^{-1}: \nattonat$ is computable.
  \begin{proof}
    Since $f$ is computable, exists $e \in \nat$ s.t. $\varphi_e =
    f$. Now is sufficent to observe that
    \[
      \begin{split}
        f^{-1}(x) &= (\mu z \; . \;  "S(e, (z)_1, x, (z)_2)")_1 \\
        &= (\mu z \; . \; |\chi_S(e, (z)_1, x, (z)_2) - 1|)_1
      \end{split}
    \]
  \end{proof}
\end{theorem}

We can now find new uncomputable functions and undecidable predicates:

\begin{theorem}
  The statement ``$\varphi_x$ is total'' is undecidable
  \begin{proof}
    Let $Tot(x)$ be the predicate
    \[ Tot(x) \equiv \mbox{ ``$\varphi_x$ is total''} \] and assume
    that it si decidable. Then the charateristic function
    \[
      \chi_{Tot}(x) = \begin{cases}
        1 & \varphi_x \mbox{ is total} \\
        0 & \mbox{otherwise}
      \end{cases}
    \]
    is computable. If this is the case, then the function
    \[
      g(x) = \begin{cases}
        \varphi_x (x) + 1 & \varphi_x \mbox{ total}\\
        1 & \mbox{otherwise}
      \end{cases}
    \]
    is both total and computable, since by hypotesis ``$\varphi_x$
    total'' is decidable and $\varphi_x(x) + 1 = \Psi(x,x) + 1$ is
    computable. But by looking at the definition by cases that it is
    false. It required the used functions to be total:
    \[ g(x) \neq (\Psi(x,x) + 1) \cdot \chi_{Tot}(x) + 0 \cdot (1
      \dotdiv \chi_{Tot}(x))\] wich is undefined if
    $\Psi(x,x) \uparrow$. The theorem of definition by cases continues
    to be valid for non total functions, but the proof must be
    changed. More in detail:
    \[
      \begin{split}
        g(x) &= (\mu z \;.\; \mbox{``}S(x,x,(z)_1, (z)_2) \wedge \lnot Tot(x)\mbox{''})_1 + 1 \\
        &= (\mu z \;.\; | \chi_{S(x,x,(z)_1, (z)_2) \wedge \lnot Tot(x)} -  1|)_1 + 1 \\
      \end{split}
    \]
    and $\forall x$ if $\varphi_x$ total
    $\Rightarrow g(x) = \varphi_x(x) + 1 \Rightarrow \varphi_x \neq
    g$. That is to say that $g$ is total and computa but different
    from every other total computable function, wich is
    absurd. Therefore $Tot(x)$ is not computable.
  \end{proof}
\end{theorem}

\textbf{Observation:} The ame applies to prove that the following
statements are undecidable (Halting problem):
\begin{itemize}
\item
  $P_1(x) \equiv \mbox{``}x \ in W_x\mbox{''} \equiv
  \mbox{``}\varphi_x(x) \downarrow \mbox{''}$
\item
  $P_2(x,y) \equiv \mbox{``} y \in W_x \mbox{''} \equiv
  \mbox{``}\varphi_x(y) \downarrow\mbox{''}$
\end{itemize}

\section{Effective operations on computable functions}
The existence of the universal function, together with the \smn
theorem allows us to prove formally the effectivity of various
operations on indices of computable functions (in other words,
programs). For example:
\begin{description}
\item[product] given $x,y$ find $\omega$ s.t.
  $\varphi_\omega = \varphi_x \cdot \varphi_y \quad(\varphi_x\cdot
  \varphi_y(z) = \varphi_x(z) \cdot \varphi_y(z))$
\item[inverse] given $x$ find $\omega$ s.t.
  $\varphi_\omega = \varphi_x^{-1}$
\end{description}
It is more or less intuitive that these operations are effective on
programs, but is less obvious how they can be proved to be computable.
\subsection{Product}
Exists a function $s: \nat^2 \rightarrow \nat$ total and computable
s.t. \[\varphi_{s(x,y)} = \varphi_x \cdot \varphi_y\]
\begin{proof}
  we define a function where x and y are arguments, and then we
  transform them into parameters.
  \[
    \begin{split}
      g(x,y,z) &= \varphi_x(z) \cdot \varphi_y(z) \\
      &= \Psi_{\mathcal{U}}(x,z) \cdot \Psi_{\mathcal{U}}(y,z)
    \end{split}
  \]
  and thats is computable by defnition. For the \smn theorem there
  exists $s: \nat^2 \rightarrow \nat$ s.t.
  \[
    \varphi_{s(x,y)}(z) = g(x,y,z) = \varphi_x(z) \cdot \varphi_y(z)
  \]
  that is to say
  \[
    \varphi_{s(x,y)} = \varphi_x \cdot \varphi_y
  \]
\end{proof}

\subsection{Squaring}
Exists $k:\nattonat$ total and computable s.t.
$\varphi_{k(x)} = \varphi_x^2$
($\forall z \quad \varphi_{k(x)}(z) = (\varphi_x(z))^2$)
\begin{proof}
$k(x) = s(x,x)$
\end{proof}

\subsection{Effectivity of recursion}
remembering the notion of primitive recursion
\[h(\vec{x}, 0) = f(\vec{x})\]
\[h(\vec{x}, y+1) = g(\vec{x}, y, f(\vec{x},y))\] and knowing that
$f,g$ are computable $\Rightarrow h$ is computable, if
$f = \varphi_{e_1}^{(k)}$ and $g = \varphi_{e_2}^{(k+2)}$ exists an
index $ind = r(e_1, e_2)$ s.t. $h = \varphi_{ind}^{(k+1)}$. We want to
prove that $r: \nat^2 \rightarrow \nat$ is total and computable. In
other words, there exists $r: \nat^2 \rightarrow \nat$ total
computable s.t. $\forall e_1,e_2$ if we define
\[h(\vec{x}, 0) = f(\vec{x})\]
\[h(\vec{x}, y+1) = g(\vec{x}, y, f(\vec{x},y))\]
then
\[h = \varphi_{r(e_1, e_2)}^{(k+1)}\]

\begin{proof}
  We just need to define
  \[
    \begin{tabu}{l l l}
      h^\prime(e_1, e_2, \vec{x}, 0) &= \varphi_{e_1}^k(\vec{x}) &= \Psi_{\mathcal{U}}^{k}(e_1, \vec{x}) \\
      h^\prime(e_1, e_2, \vec{x}, y+1) &= \varphi_{e_2}^{k+2}(\vec{x},y,f(\vec{x}, y)) &= \Psi_{\mathcal{U}}^{k+2}(e_1, e_2, \vec{x}, y, h^\prime(e_1, e_2, \vec{x} , y))
    \end{tabu}
  \]
  $H^\prime$ defined by primitive recursion from computable functions,
  is computable, and for the \smn theorem there exists
  $r: \nat^2 \rightarrow \nat$ total and computable s.t.
  \[\varphi_{r(e_1, e_2)}(\vec{x}, y) = h^\prime (e_1, e_2, \vec{x}, y)\]
  as we wanted.
\end{proof}

\subsection{Effectiveness of the inverse function}
Exists $k: \nattonat$ total and computable s.t.
\[\forall x \in \nat \quad \mbox{if } \varphi_x \mbox{ is injetive }
  \Rightarrow \varphi_{k(x)} = (\varphi_x)^{-1}\]

\begin{proof}
  we can define a function $g(x,y)$ that for those $x$
  s.t. $\varphi_x$ is injective it:
  \[
    g(x,y) = \varphi_x^{-1}(y) = \begin{cases}
      z & \exists z \mbox{ s.t. } \varphi_x(z) = y \\
      \uparrow & \mbox{otherwise}
    \end{cases}
  \]
  \[
    g(x,y) = (\mu \omega \; . \; |\chi_{S(x, (\omega)_1, y, (\omega)_2)} - 1|)_1
  \]
  and for the \smn theorem exists a $k: \nattonat$ total and
  computable s.t.
  \[\varphi_{k(x)}(y) = g(x,y) = \varphi_x^{-1}(y) \mbox{ if }
    \varphi_x \mbox{ is injective}\]
\end{proof}

\section{manipulating domains and codomains}
(a) Exists a total computable function $S : \nat^2 \rightarrow \nat$
s.t. \[W_{S(x,y)} = W_x \cup W_y\]
\begin{proof}
  we want a function s.t. $\varphi_{S(x,y)}(z)\downarrow$ iff
  $\varphi_x(z)\downarrow$ or $\varphi_y(z) \downarrow$. We can define
  a function with $x,y$ as arguments that capures both properties

  \[
    g(x,y,z) = \begin{cases}
      0 & z \in W_z \mbox{ or } z \in W_y \\
      \uparrow  & \mbox{otherwise}
    \end{cases}
  \]
  wich is computable:
  \[
    g(x,y,z) = \underline{0}(\mu \omega \; . \; |\chi_{H(x,z,\omega)
      \wedge H(y,z,\omega)} - 1|)
  \]
  \textbf{Note:} one $\omega$ is enough, since
  $\exists \omega_1 \; . \; P_1(\omega_1) \wedge \exists \omega_2 \;
  . \; P_2(\omega_2) \equiv \exists \omega \; . \; P_1(\omega) \wedge
  P_2(\omega)$

  So, for the \smn theorem exists $s:\nat^2 \rightarrow \nat$
  computable and total s.t. \[\varphi_{S(x,y)}(z) = g(x,y,z)\]
\end{proof}

(b) Exists $k:\nat^2 \rightarrow \nat$ computable and total s.t.
\[\forall x,y \quad E_{K(x,y)} = E_x \cup E_y\]
\begin{proof}
  We want the value of $\varphi_{S(x,y)}$ to be the same of the
  functions $\varphi_x and \varphi_y$. To do this we can simulate
  $\varphi_x$ on even numbers and $\varphi_y$ on odd numbers. We
  define a function where $x$ and $y$ are arguments
  \[
    g(x,y,z) = \begin{cases}
      \varphi_x(\frac{z}{2}) & \mbox{if } z \mbox{ even} \\
      \varphi_y(\frac{z-1}{2}) & \mbox{if } z \mbox{ odd}
    \end{cases}
  \]
  computable since
  \begin{multline*}
    g(x,y,z) = (\mu \omega \; . \; |\max\{\chi_S(x,qt(2,z),(\omega)_1,(\omega)_2) \; \cdot \; \overline{sg}(rm(2,z)), \\ \chi_S(y,qt(2,z),(\omega)_1, (\omega)_2) \; \cdot \; sg(rm(2,z))\}- 1|)_1
  \end{multline*}
  And for the \smn theorem exists $k: \nat^2 \rightarrow \nat$
  computable and total s.t. \[\varphi_{K(x,y)}(z) = g(x,y,z)\]
  So
  \[
    \begin{split}
      \nu \in E_{K(x,y)} & \Leftrightarrow \exists z \; . \; \varphi_{S(x,y)}(z) = g(x,y,z) = \nu \\
      & \Leftrightarrow \exists z \; . \; \begin{cases}
        z \mbox{ even and } & \varphi_x(\frac{z}{2}) = \nu \\
        z \mbox{ odd and } & \varphi_y(\frac{z-1}{2}) = \nu \\
      \end{cases} \\
      & \Leftrightarrow \exists z \; . \; \varphi_x(z) = \nu \wedge
      \varphi_y(z) = \nu \Leftrightarrow \omega \in E_x \cup E_y
    \end{split}
  \]
\end{proof}

(c) Exists $k : \nattonat$ computable and total s.t. $E_{k(x)} = W_x$
\begin{proof}
  We want
  \[(y \in W_x \Leftrightarrow y \in E_{k(x)}) \equiv
    (\varphi_x(y)\downarrow \Leftrightarrow \exists z \; . \;
    \varphi_{k(x)}(z) = y) \]
  and we can define
  \[
    \begin{split}
      g(x,y) &= \begin{cases}
        y & y \in W_x \\
        \uparrow & \mbox{otherwise}
      \end{cases} \\
      &= \mathds{1}( \univ (x,y)) \cdot y
    \end{split}
  \]
  wich is computable, so for the \smn theorem exists $k : \nattonat$
  computable and total s.t. \[\varphi_{k(x)} = g(x,y)\]
  in other words
  \[y \in E_{k(x)} \Leftrightarrow \varphi_{k(x)}(y) = y
    \Leftrightarrow g(x,y) = y  \Leftrightarrow y \in W_x\]
\end{proof}

(d) Given $f : \nattonat$ computable, exists $k : \nattonat$
computable and total s.t. $\forall x \quad W_{k(x)} = f^{-1}(W_x)$
\begin{proof}
  we want a function s.t.
  \[y \in W_{k(x)} \Leftrightarrow f(y) \downarrow \mbox{ and } f(y)
    \in W_x \]
  in other words
  \[\varphi_{k(x)}(y) \downarrow \Leftrightarrow \varphi_x(f(y))
    \downarrow\]
  so we can define
  \[g(x,y) = \varphi_x(f(y)) = \univ(x, f(y))\] computable by
  definition. So for the \smn theorem exists $k: \nattonat$ computable
  and total s.t. \(\varphi_{k(x)}(y) = g(x,y)\). So
  \[
    \begin{split}
      y \in W_{k(x)} & \Leftrightarrow \varphi_{k(x)}(y) = g(x,y) = \varphi_x(f(y)) \downarrow \\
      & \Leftrightarrow f(y)\downarrow \mbox{ and } f(y) \in W_x \\
      & \Leftrightarrow y \in f^{-1}(W_x)
    \end{split}
  \]
\end{proof}

\section{Operations on predicates}
Exists $k : \nattonat$ computable and total s.t. if
$\varphi_x = \chi_a$ is the charateristic function of a decidable
predicate $Q$, then $\varphi_{k(x)} = \chi_{\neg Q}$
\begin{proof}
  we can define \[g(x,y) = 1 \dotdiv \varphi_x(y) = 1 - \univ(x,y) \]
  wich is computable by deifinition. So, for the \smn theorem exists
  $k$ computable and total s.t. \[g(x,y) = \varphi_{k(x)}\] this way
  if $\varphi_x = \chi_Q$
  \[
    g(x,y) = 1-\varphi_x(y) = \varphi_{k(x)}(y) = 1 \Leftrightarrow
    \varphi_x(y) = 0 \Leftrightarrow \chi_Q(y) = 0
  \]
  therefore
  \[
    \varphi_{k(x)} = \chi_{\neg Q}
  \]
\end{proof}