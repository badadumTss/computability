\chapter {Universal Function}
We'll now see how the theory we described up until now allows us to
prove something a bit surprising. This is to say, the existence of
universal functions/programs, able to reproduce the behaviour of each
and every other computable function/program. Let's consider the function

\[
  \psi(x,y) = \varphi_x(y)
\]

We can already observe that $\psi$ is a ``universal function'', that
captures all unary functions $\varphi_1, \varphi_2, \dots$. In fact,
for all fixed $e \in \nat$
\[
  g(y) = \psi(e,y) = \varphi_e(y) \quad \rightsquigarrow \quad g = \varphi_e
\]
so, thanks to $e$, $\psi$ rapresents all computable functions in the
form $\nattonat$.

\begin{definition}
  The universal function for k-ary functions (with $k \in \nat$) is
  defined as
  \[
    \psi_{\mathcal{U}}^{(k)} : \nat^{k+1} \rightarrow \nat
  \]
  \[
    \psi_{\mathcal{U}}^{(k)} (e, \vec{x}) = \varphi_e^{(k)}(\vec{x})
  \]
\end{definition}
$\psi$ functions are this well defined, but are they computable? If
the answer is yes a program $P_{\mathcal{U}}$ computing $\psi$ would
be able to compute all k-ary functions. It includes in itself all
other computable programs, and we can call it a Universal Computer
\cite{davis:2011}.

At first glance this might seem odd, but tinking again it recives in
input:
\begin{itemize}
\item $e$ (the index of the program, a \textit{description} of the
  program $P_e$ to run)
\item $\vec{x}$ the arguments
\end{itemize}
We observe that it fits the description of an interpreter.

In fact, the following result applies

\begin{theorem}
  $\forall k \geq 1$ the universal function $\psi_{\mathcal{U}}^{(k)}$
  is computable.

  \begin{proof}
    
  \end{proof}
\end{theorem}

% Program that takes input $ e $ and $ \vec{x} $ and returns
% $ \phi_e(\vec{x}) \forall e$ so I find all the computable functions.

% More specifically, the function $ \psi_u(x,y) = \phi_x(y)$,
% $ \psi : \nat^2 \rightarrow \nat $

% In other words, the interpreter exists.

% \textbf{Theorem}: This is computable.
% $ \forall k \geq 1 \quad \psi_u^{(k)} $ is computable.

% Let's prove it. Suppose we have a fixed $ k \geq 1 $.

% How do I calculate
% $ e \in \nat, \vec{x} \in \nat^k, \psi_u^{k}(e, \vec{x}) $?

% Fact $e$, calculation $ \gamma^{-1}(e) = P_e $

% But we don't want to decode the program.
