\chapter{Recursive sets}
In previous chapters, we saw many computable functions and decidable problems, but
only in few cases we gave examples of the large class of
non-computable functions and undecidable predicates. For this reason,
we want to start a mathematical study of:
\begin{itemize}
\item classes of undecidable predicates
\item techniques to prove the undecidability of some predicates
\end{itemize}
This way we can highlight the limits of computers abilities and give a
structure to problems classes (spoiler: the majority of interesting
predicates are undecidable).

We'll focus on \emph{numerical sets} $X \subseteq \nat$ and try to find
an answer to the problem ``how difficult it is to determine weather
$x \in X$''? This way we'll find
\begin{itemize}
\item recursive sets
\item recursively enumerable sets
\end{itemize}

\section{Recursive sets}
\begin{definition}
  A set $A \subseteq \nat$ is \emph{recursive} if its characteristic
  function
  \begin{gather*}
    \chi_A : \nattonat \\
    \chi_A(x) = \begin{cases}
      1 & x\in A \\
      0 & x \notin A
    \end{cases}
  \end{gather*}
  is computable.
\end{definition}

(In other words, if the predicate ``$x \in A$'' is decidable.)

\textbf{Note:}
\begin{itemize}
\item if $\chi_A \in \pr$ we'll say that $A$ is \emph{primitively}
  recursive.
\item the notion can be extended to subsets of $\nat^k$, but we'll
  stick to $\nat$ subsets, since every $\nat^k$ subset can be encoded
  into $\nat$
\end{itemize}

\paragraph{Examples}
These are recursive:
\begin{enumerate}[label=(\alph*)]
\item $\nat$, since $\chi_\nat (x) = 1$ is computable;
\item prime numbers
  \[
    Pr(x) = \begin{cases}
      1 & \mbox{if $x$ is prime} \\
      0 & \mbox{otherwise}
    \end{cases}
  \]
  is computable;
\item each and every finite set. In fact, given $A \subset \nat$ with
  $|A| < \infty \quad A = \{x_1, x_2, \dots, x_n\}$ we have that
  \[
    \chi_A(x) = \overline{sg}\left( \prod_{i=1}^n|x - x_i| \right)
  \]
  is computable
\end{enumerate}

On the other hand, these are definitely not recursive:
\begin{enumerate}[label=(\alph*)]
\item $k = \left\{ x \; | \; x \in W_x \right\} $, since
  \[
    \chi){k(x)} = \begin{cases}
      1 & x \in W_x \\
      0 & x \notin W_x
    \end{cases}
  \]
  is not computable;
\item $\left\{ x \; | \; \varphi_x \mbox{ total } \right\} $
\end{enumerate}

\mbox{\\}

\paragraph{Closure of recursive sets}
If $A,B \subset \nat$ are recursive, then they are also recursive
\begin{enumerate}[label=\arabic*)]
\item $\overline{A} = \nat - A$
\item $A \cap B$
\item $A \cup B$
\end{enumerate}

\subsection{Reduction process}
This process is a simple but very important process to study decidability
problems. It basically tries to formalize the intuition of a problem
$\mathcal{A}$ being ``easier'' then another one, $\mathcal{B}$

\newcommand{\red}{\ensuremath{\leq_m}}
\begin{definition}
  Let $A,B \subseteq \nat$. We say that the problem $x \in A$ is
  \emph{reducible} to the problem $x \in B$ (or also that directly $A$
  is reducible to $B$), and we write $A \red B$ if exists
  $f : \nattonat$ computable and total s.t.
  \[x \in A \quad  \Leftrightarrow \quad f(x) \in B\]
\end{definition}
In this case, $f$ is the \emph{reduction function}.

\textbf{Proposition:} Let $A,B \subseteq \nat$ s.t. $A \red B$ then
\begin{enumerate}[label=\arabic*)]
\item if $B$ is recursive, then $A$ is recursive
\item of $A$ is not recursive, then $B$ is not recursive
\end{enumerate}

\begin{proof}
We just need to observe that $\chi_A = \chi_B \circ f$
\end{proof}

We know that $k = \{ x | x \in W_x \}$ is not recursive. Let's see how
the recursiveness of other sets can be proven by reduction to this
set which we know for certain is not recursive.
\begin{example}[$k \leq \{x | \varphi_x $ total $ \}$ ]
  \begin{proof}
    we need to prove that there exists $f : \nattonat$ computable and total
    s.t. $x \in k \Leftrightarrow f(x) \in A$. In other words
    \[ x\in W_x \Leftrightarrow  \varphi_{f(x)} \mbox{ is total} \]
    To do so we can define
    \[
      g(x,y) = \begin{cases}
        1 & x\in W_x \\
        \uparrow & \mbox{otherwise}
      \end{cases}
    \]
    which is computable. This fact is easily proven by rewriting it as
    \[
      g(x,y) = \mathds{1}(\varphi_x(x)) = \mathds{1}(\univ(x,x))
    \]
    then, for the \smn theorem we have that it exists $s: \nattonat$
    computable and total s.t. \[\varphi_{s(x)}(y) = g(x,y)\] and $s$
    is the very function we were searching for; in fact
    \[x \in k \Rightarrow x \in W_x \Rightarrow \forall y
      \varphi_{s(x)}(y) = g(x,y) = 1 \Rightarrow \varphi_{s(x)} \mbox{
        total } \Rightarrow s(x) \in A\]
    \[x \notin k \Rightarrow x \notin W_x \Rightarrow \forall y
      \varphi_{s(x)}(y) = g(x,y) = \uparrow \Rightarrow \varphi_{s(x)}
      \mbox{ non total } \Rightarrow s(x) \notin A\]
  \end{proof}
\end{example}

\begin{example}[$\{(x,y) | y \in W_x\}$]
  To see it as a subset of $\nat$ we just need to encode couples as
  numbers
  \[B = \{ n | \Pi_2(n) \in W_{\Pi_1(n)}\}\]

  \paragraph{$K \leq_m B$} We need to find a function $f : \nattonat$
  computable and total s.t. \(n \in k \Leftrightarrow f(n) \in B\). To
  do this we just need to consider \[f(n) = \Pi(n,n)\] We will have
  that
  \[
    \begin{tabu}{c c c c c c c}
      n \in k & \Rightarrow & n \in W_n & \Rightarrow & Pi_2(f(n)) \in W_{\Pi_1(f(n))} & \Rightarrow f(n) \in B \\
      & \Leftarrow & & \Leftarrow & & \Leftarrow &

    \end{tabu}
  \]

  \paragraph{$B \leq_m k$}
  We can define
  \[
    g(x,y,z) = \begin{cases}
      1 & y \in W_x \\
      \uparrow & \mbox{otherwise}
    \end{cases}
  \]
  is computable, in fact
  \[
    g(x,y,z) = \mathds{1}(\varphi_x(y)) = \mathds{1}(\univ (x,y))
  \]
  therefore, for the \smn theorem
  $\exists s : \nat^2 \rightarrow \nat$ computable and total s.t.
  \[
    g(x,y,z) = \varphi_{s(x,y)}(z)
  \]
  and the reduction function will be \[f(n) = s(Pi_1(n), \Pi_2(n))\]
\end{example}

\begin{example}[Input problem]
  prove that $\forall n \in \nat$
  \[
    A_n = \{x  | \varphi_x (n) \downarrow\}
  \]
  is not recursive.

  \begin{proof}
    we will prove that $K \leq A_n$. We have to define a function $f$
    s.t.
    \begin{gather*}
      x \in k \Leftrightarrow f(x) \in A_n \\
      x \in W_x \Leftrightarrow \varphi_{f(x)}(n) \downarrow
    \end{gather*}
    For example
    \[
      \begin{split}
        g(x,y) &= \begin{cases}
          1 & x \in W_x \\
          \uparrow & \mbox{otherwise}
        \end{cases} \\
        &= \mathds{1}(\univ(x,x))
      \end{split}
    \]
    which is computable, and therefore for the \smn theorem exists
    $f: \nattonat$ computable and total s.t.
    $g(x,y) = \varphi_{f(x)}(y)$, and so
    \begin{gather*}
      x \in k \Rightarrow f(x) \in A_n \\
      x \notin  k \Rightarrow f(x) \notin A_n
    \end{gather*}
  \end{proof}
\end{example}

\begin{example}[The output problem]
  Since $n \in \nat \quad B_n\{ x | n \in E_x\}$ is not recursive
  \begin{proof}
    we'll prove that $k \leq_m B_n$
    \[
      \begin{split}
        g(x,y) &= \begin{cases}
          n & x \in W_x \\
          \uparrow & \mbox{otherwise}
        \end{cases} \\
        &= n \cdot \mathds{1}(\univ(x,x))
      \end{split}
    \]
    computable, and therefore for the \smn theorem exists a function
    $s : \nattonat$ s.t.
    \[
      \forall y \quad g(x,y) = \varphi_{s(x)}(y)
    \]
  \end{proof}
  moreover
  \begin{gather*}
    x \in k \Rightarrow s(x) \in B(n) \\
    x \notin k \Rightarrow s(x) \notin B_n
  \end{gather*}
\end{example}

\textbf{Observation:} Let $A,B \subseteq \nat$ with $A \leq_m B$
through a reduction function $f : \nattonat$ (total and computable)
injective. One could think that since $f^{-1}$ is computable, then
also $B \leq_m A$. In reality this does not happen, since $f^{-1}$ is
not total and so reduces $A$ to a ``subproblem'' of $B$ even though
it has no complexity relationships with $B$ in principle.
