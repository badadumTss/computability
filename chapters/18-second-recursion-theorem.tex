\chapter{Second recursion theorem}
Let h be total, computable and extensional function. Then, for the
theorem (\ref{th:myhill-shepherdson2}) there exists only one recursive
functional $\Phi$ such that
\[ \forall e \quad \Phi(\varphi_e) = \varphi_{h(e)} \] Since $\Phi$ is
recursive, for the first recursion theorem (\ref{th:first-recursion})
it has a minimum fixed point $f_\Phi$ computable, and therefore exists
$e\in \nat$ s.t. $f_\Phi = \varphi$. This means that:
\[
  \varphi_e = \Phi(\varphi_e) = \varphi_{h(e)}
\]
That is to say that if $h$ is computable total and extensional then it
exists $n$ s.t. \[\varphi_n = \varphi_{h(n)}\] The second recursion
theorem states that this is valid also when h is not extensional.  This
means that a function $h : \nattonat$ cannot be thought as a
functional on computable functions. Instead, we can think that it is a
\emph{function} that is defined on \emph{programs}.

\begin{theorem}[Second recursion theorem]\label{th:second-recursion}
  If $h : \nattonat$ a function computable and total, then there exists
  $n \in \nat$ such that:
  \[
    \varphi_{h(n)} = \varphi_n
  \]
  \begin{proof}
    Lets consider the function
    $g(x,y) = \varphi_{h(\varphi_x(x))}(y)$. It is obviously
    computable:
    \[ g(x,y) = \univ(h(\univ(x,x)), y) \] This means that for the
    \smn theorem exists $S : \nattonat$ computable and total s.t.
    \[
      g(x,y) = \varphi_{S(x)}(y) \quad \forall y
    \]
    Since $S$ is computable $\exists m$ s.t. $S = \varphi_m$ and therefore
    we can substitute the variables
    \[
      \varphi_{h(\varphi_x(x))}(y) = \varphi_{\varphi_m(x)}(y) \quad \forall y
    \]
    consequently, if $m=x$
    \[
      \varphi_{h(\varphi_m(m))}(y) = \varphi_{\varphi_m(m)}(y) \quad \forall y
    \]
    and if we consider $n = \varphi_m(m)$ (which is defined since
    $S = \varphi_m$ is total)
    \[
      \varphi_{h(n)}(y)  = \varphi_n(y) \quad \forall y
    \]
    that is to say
    \[
      \varphi_{h(n)} = \varphi_n
    \]
  \end{proof}
\end{theorem}

This theorem can therefore be interpreted in the following manner:
``given any effective procedure to transform programs, there exists a
program such that when properly modified does exactly what it did
before''.

Or again ``it is impossible to write a program that edits the core of
all programs''.

\textbf{Note:} the proof can appear mysterious, but after a closer
inspection, it clearly appears to be a simple diagonalization.

\textbf{Note again:} the result of this theorem is extremely
deep. This way, many theorems we've seen up until now are just
corollaries of it:

\begin{corollary}[Rice theorem]
  Let $\emptyset \neq A \subsetneq \nat$. If $A$ is saturated, then
  $A$ is not rich.
  \begin{proof}
    Let $\emptyset \subsetneq A \subsetneq \nat$ and suppose that $A$
    (and therefore $\bar{A}$) is recursive. This means that
    $\chi_A, \chi_{\bar{A}}$ are computable. Now, let
    $a_0 \in \bar{A}, a_1 \in A$ be fixed elements (they clearly
    exist, since A is not trivial). We can define
    \[
      f(x) = \begin{cases}
        a_0 & x \in A \\
        a_1 & x \in \bar{A}
      \end{cases} = a_0\chi_A(x) + a_1\chi_{\bar{A}}(x)
    \]
    it's  easy to see that
    \begin{equation}\label{eq:xinafxinbara}
      x\in A \mbox{ iff } f(x) \in \bar{A}
    \end{equation}
    Since $f$ is computable and total, for the second recursion
    theorem (\ref{th:second-recursion}) $\exists n$ s.t.
    $\varphi_n = \varphi_{f(n)}$ and $A$ is saturated, which means that
    either $n, f(n) \in A$ or $n, f(n) \in \bar{A}$, which contradicts
    \ref{eq:xinafxinbara}.
  \end{proof}
\end{corollary}

\begin{corollary}
  $K$ is not saturated
  \begin{proof}
    Let $k = \{ x \mid x \in W_x\}$ recursive for the sake of the
    argument. and let $e_0, e_1$ be indexes s.t.
    $\varphi_{e_0} = \emptyset$ and
    $\varphi_{e_1} = \lambda x \; . \; x$. We can define
    \[
      f(x) = \begin{cases}
        e_0 & x \in K \\
        e_0 & x \notin K
      \end{cases}
    \]
    $f$ is both computable and total, therefore according to
    (\ref{th:second-recursion})
    $\exists n \; . \; \varphi_n = \varphi_{f(n)}$, but:
    \[
      \begin{tabu}{l l l l l l l}
        n \in K & \Rightarrow & \varphi_n = \varphi_{f(n)} = \varphi_{e_0} & \Rightarrow & \varphi_n(n) = \varphi_{e_0} (n) \uparrow & \Rightarrow & n \notin K \\
        n \notin K & \Rightarrow & \varphi_n = \varphi_{f(n)} = \varphi_{e_1} & \Rightarrow & \varphi_n(n) = \varphi_{e_1} (n) = n & \Rightarrow & n \in K \\
      \end{tabu}
    \]
    which is absurd.
  \end{proof}
\end{corollary}

\begin{corollary}
  $K$ is not saturated
  \begin{proof}
    Let $n_0$ s.t. $\varphi_{n_0} = \{(n_0, n_0)\}$. We know that
    there are infinitely many indices for the same function; so let
    $n \neq n_0$ s.t. $\varphi_n = \varphi_{n_0}$.
    \[
      \varphi_n(n) \uparrow \Rightarrow n \notin K
    \]
  \end{proof}
\end{corollary}
